\documentclass[a4paper,10pt]{article}
%%%%%%%%%%%%%%%%%%%%%%%%%%%%%%%%%%%%%%%%%%%%%%%%%%%%%%%%%%%%%%%%%%%%%%%%%%%%%%%%%%%%%%%%%%%%%%%%%%%%%%%%%%%%%%%%%%%%%%%%%%%%%%%%%%%%%%%%%%%%%%%%%%%%%%%%%%%%%%%%%%%%%%%%%%%%%%%%%%%%%%%%%%%%%%%%%%%%%%%%%%%%%%%%%%%%%%%%%%%%%%%%%%%%%%%%%%%%%%%%%%%%%%%%%%%%
%\usepackage[tbtags]{amsmath}
%\usepackage{graphicx,amssymb,amsfonts,amsthm}
%\usepackage{setspace}
%\usepackage{nopageno}
%\usepackage{natbib}
\usepackage{graphicx,amssymb,amsfonts,amsthm}
\usepackage[tbtags]{amsmath}
\usepackage{natbib,upgreek,mathtools}
\usepackage{rotating, caption}
%\usepackage{endfloat}
\usepackage{tgpagella}
\usepackage{xcolor}
\usepackage{hyperref}

%%%%%%%%%%%%%%%%%%%%%%%%%%%%%%%%%%%%%%%%%%%%%%%%%%%%%%%%%%%%
\definecolor{MyChange}{rgb}{1,0,0}
%%%%%%%%%%%%%%%%%%%%%%%%%%%%%%%%%%%%%%%%%%%%%%%%%%%%%%%%%%%%

%\usepackage[nolists]{endfloat}

%%%%%%%%%%%%%%%%%%%%%%%%%%%%%%%%%%%%%%%%%%%%%%%%%%%%%%%%%%%%
%Notation
%
\newcommand{\pbi}{\begin{itemize}}
	\newcommand{\pei}{\end{itemize}}
\newcommand{\pii}{\item}
%
\newcommand{\pbc}{\begin{center}}
	\newcommand{\pec}{\end{center}}
%
\newcommand{\pbe}{\begin{eqnarray*}}
	\newcommand{\pee}{\end{eqnarray*}}
%
\providecommand{\Pr}{\mathbb{Pr}} %real numbers
\let\hat=\widehat
\let\geq=\geqslant
\let\leq=\leqslant
%
\newcommand{\defeq}{\,\stackrel{{\rm \vartriangle}}{=}\,}
\newcommand{\pms}{\quad}
\newcommand{\simindep}{\,\stackrel{{\rm indep}}{\sim}\,}
\DeclareMathOperator*{\argmin}{arg\,\min}
%\newcommand{\med}{#1}{\,\stackrel{{\rm median}}{#1}\,}
%\newcommand\med[1]{\stackrel{{\rm median}}{#1}}
%
%\newcommand\med[1]{\stackrel{\rm median}{{\normalfont\mbox{#1}}}}
\newcommand\med[1]{\underset{#1}{\mathrm{med}}}

\providecommand{\Xd}{\dot{X}}
\providecommand{\zd}{\dot{z}}
\providecommand{\yd}{\dot{y}}
\providecommand{\xd}{\dot{x}}
\providecommand{\thetad}{\dot{\theta}}
\providecommand{\phid}{\dot{\phi}}
\providecommand{\nd}{{\dot{n}}}

\providecommand{\rhot}{\tilde{\rho}}
\providecommand{\sigmat}{\tilde{\sigma}}
\providecommand{\xit}{\tilde{\xi}}
\providecommand{\ut}{{\tilde{u}}}
\providecommand{\qt}{{\tilde{q}}}
\providecommand{\Qt}{{\tilde{Q}}}
\providecommand{\qb}{{\breve{q}}}
\providecommand{\Qb}{{\breve{Q}}}
%
\providecommand{\eps}{\epsilon}
\providecommand{\cvr}{{\theta,\phi}}
\providecommand{\cvrA}{{\theta \text{ and } \phi}}
\providecommand{\prm}{\alpha, \gamma, \sigma, \xi}
\providecommand{\prmA}{\alpha, \gamma, \sigma \text{ and } \xi}
\providecommand{\prmn}{\alpha, \gamma, \nu, \xi} %uses nu
\providecommand{\prmAn}{\alpha, \gamma, \nu \text{ and } \xi} %uses nu
%
\newcommand{\GP}{\mathrm{GP}}
\newcommand{\W}{\mathrm{Wbl}}
\newcommand{\TW}{\mathrm{TW}}
%
\newcommand{\un}[1]{\boldsymbol{#1}}
%
\providecommand{\np}{\vspace{10pt}}
%
\newcommand{\ed}[1]{\textcolor{red}{#1}}

%\onehalfspacing
\setlength{\parindent}{0cm}
\setlength{\parskip}{1em}

\providecommand{\np}{\vspace{10pt}}

%Page size
\oddsidemargin  -0.7in
\evensidemargin -0.7in
\textwidth      7.6in
\headheight     0.0in
\topmargin      -1.0in
\textheight     10.5in

\begin{document}
	
	\title{\textbf{Reply on first review} \np \\ {XXX} \np \\ by Leach et al.}
	\np
	\author{Please address all correspondence to Philip Jonathan (\emph{p.jonathan@lancaster.ac.uk})}
	\date{}
	\maketitle
	
	\section*{Summary}
	
	We thank four reviewers (referenced Reviewer 2-5) sincerely for their time and expertise in reviewing the manuscript. We note that both reviewers find merit in the work. 
	
	Reviewer 2 XXX. Reviewer 3 XXX. Reviewer 4 XXX. Reviewer 5 XXX.
	
	More generally, both reviewers make suggestions for improvement of the manuscript, which we reply to below, entailing some additions and modification of the manuscript. Where we feel the manuscript is already adequate on a point, we provide a note of explanation. 
	
	In the following sections, reviewer comments on the manuscript are given verbatim in italics and referenced by review section number `S', review reference number `R', comment number `C'. Our author replies (suffixed `A') are given in normal font per review comment. Larger changes to the manuscript are given in \ed{red} in both rejoinder and revised manuscript. References to locations in the article are made by us per section (e.g Section~4, S3.5), to avoid confusion with page- and line-number differences between manuscripts. 
	
	We hope that the revised manuscript is acceptable for publication, but would be happy of course to revise further in light of additional review comments.
	
	\section*{Review section 1. Are the objectives and the rationale of the study clearly stated?}
	\emph{S1.R2.C0 General comment. This study provides a detailed analysis of CMIP6 global coupled models, focusing on the effects of climate change at site-specific, regional, and global scales. The examination of changes in extreme quantiles, particularly the 100-year return value of climate variables, is highly relevant and enhances understanding of long-term trends. The methodology, combining extreme value analysis (GEV) and non-homogeneous Gaussian regression (NHGR), is well-suited for quantifying extreme events and spatial trends.The structure of the paper is clearly laid out, with a logical progression from data description and methodology to the presentation of results. The separation of global, climate zone, and specific location analyses provides a thorough exploration of the different scales at which climate change impacts may manifest. The inclusion of supplementary material further enhances the accessibility and understanding of the data and results. Only minor revisions are required to further enhance the clarity and impact of the paper. These include making a few refinements to the text for improved flow and precision, correcting typographical errors, and expanding the discussion of the results. Once these minor adjustments have been made, I believe the paper will be ready for publication. Overall, this work represents a significant contribution to the field, with well-defined objectives and a robust methodology. It offers valuable insights into the future projections of key climate variables and highlights the importance of considering both extremes and spatial averages when assessing climate change impacts.}
	
	S1.R2.C0.A1 We thank the reviewer for the positive comments. We have accommodated the majority of reviewer suggestions in the revised article, details below. XXX
		
	\emph{S1.R2.C1 Remark \#1 The introduction is thoughtfully organised, drawing on a range of references, including recent ones. The objectives are clearly articulated, and the article’s structure is comprehensively outlined. Furthermore, supplementary material is included to reinforce and support the content of the article. However, the data referenced in Leach (2024) is not available, as the provided link is not working. Please review and update the link in the reference “Leach, C., 2024. Changes over time in the 100-year return value of climate model variables: data. https://github.com/Callum-Leach/Climate-Change-100-Year-Return-Value-Offshore-Engineering-Data”, accordingly.}
	
	S1.R2.C1.A1 The hyperlink is correct, but without the double quotation at the end of the string. We have ensured that the double quotation mark is clearly separated from the link text. XXX
		
	\emph{S1.R2.C2 Remark \#2 In the first sentence of the 3rd paragraph of Section “2. Global coupled model output” (page 3 of the manuscript), it would be helpful to explicitly include the corresponding letters for initialisation, physics, and forcing of the model when describing the ensemble members. For improved clarity, you might consider revising the sentence as follows: “(…) we examine output for five climate model ensemble members (rX) where available; these correspond to a common initialisation (iX), physics (pX) and forcing (fX) per GCM, (…)”. This would provide readers with a clearer understanding of the specific configuration of each ensemble member, particularly when these details are referenced in Table 1.}
	
	S1.R2.C2.A1 We agree, and have added the description as suggested. XXX
	
	\emph{S1.R2.C3 Remark \#3 It would be beneficial to explicitly highlight why the 100-year return value is a critical metric for understanding climate change impacts. This could be incorporated into the introduction to ensure better alignment with the conclusions.}
	
	S1.R2.C3.A1 We agree, and have added a description as suggested. XXX

	\emph{S1.R2.C4 Remark \#4 It would be helpful to clarify how this study contributes to bridging gaps in the existing literature, thereby strengthening the rationale for the research. This could be highlighted in the introduction or the discussion section for better context.}
	
	S1.R2.C4.A1 We agree, and have added a description as suggested. XXX

	\emph{S1.R3.C0 yes}

	\emph{S1.R4.C0 yes}

	\emph{S1.R5.C0 Generally yes, however, the choices of variables and models needs to be put into context. Please see the Reviewer Comments to the Author Section.}
	
	S1.R2.C5.A1 Refer to S10.R5.CXXX.
	
	%Section 2%
	\section*{Review section 2. If applicable, is the application/theory/method/study reported in sufficient detail to allow for its replicability and/or reproducibility?}
	\emph{S2.R2.C0 Yes [X] No [] N/A []}
	
	S2.R2.C0.A1 XXX
		
	\emph{S2.R2.C1 Remark \#5 It would be constructive to clarify in Section “3. Methodology" the rationale or criteria used to select and present the results from the UKESM1-0-LL GCM in Section "SM5 Diagnostic plots for GEVR model fitting to global annual data, and NHGR modelling fitting to global means, for UKESM1-0-LL" of the Supplementary Material, as opposed to results from other models.}
	
	S2.R2.C1.A1 Thanks. We chose UKESM1-0-LL as a typical example. We have added text to this effect in SXXX of the revised manuscript. XXX
		
	\emph{S2.R2.C2 Remark \#6 It would be insightful to provide additional information about the specific parameters and settings used in the non-stationary GEV and NHGR models, as this would enable other researchers to replicate the statistical analysis.}
	
	S2.R2.C2.A1 XXX (Reference appendix and other sections where the model setup was discussed).
	
	\emph{S2.R3.C0 Yes [X] No [] N/A []}

	\emph{S2.R4.C0 Yes [X] No [] N/A []}
	
	\emph{S2.R5.C0 Yes [X] No [] N/A []}

	%Section 10%
	\section*{Review section 10}
	\emph{S10.R2.C0 XXX}
	
	S10.R2.C0.A1 XXX
		
	\emph{S10.R3.C0 XXX}
	
	S10.R3.C0.A1 XXX
		
	\emph{S10.R4.C0 XXX}
	
	S10.R4.C0.A1 XXX
	
	\emph{S10.R5.C0 XXX}
	
	S10.R5.C0.A1 XXX


	%\pagebreak
	\bibliographystyle{elsarticle-harv}
	\bibliography{phil}
	%\bibliography{C:/Philip/LaTeX/Bibliography/phil}
	%\bibliography{C:/Users/Philip.Jonathan/PhilipGit/Code/LaTeX/phil}
	%\bibliography{C:/Philip/Git/Cod/LaTeX/phil}
	%\bibliography{phil}
	
\end{document}