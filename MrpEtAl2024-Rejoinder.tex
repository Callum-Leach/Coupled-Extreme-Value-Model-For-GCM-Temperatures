\documentclass[a4paper,10pt]{article}
%%%%%%%%%%%%%%%%%%%%%%%%%%%%%%%%%%%%%%%%%%%%%%%%%%%%%%%%%%%%%%%%%%%%%%%%%%%%%%%%%%%%%%%%%%%%%%%%%%%%%%%%%%%%%%%%%%%%%%%%%%%%%%%%%%%%%%%%%%%%%%%%%%%%%%%%%%%%%%%%%%%%%%%%%%%%%%%%%%%%%%%%%%%%%%%%%%%%%%%%%%%%%%%%%%%%%%%%%%%%%%%%%%%%%%%%%%%%%%%%%%%%%%%%%%%%
%\usepackage[tbtags]{amsmath}
%\usepackage{graphicx,amssymb,amsfonts,amsthm}
%\usepackage{setspace}
%\usepackage{nopageno}
%\usepackage{natbib}
\usepackage{graphicx,amssymb,amsfonts,amsthm}
\usepackage[tbtags]{amsmath}
\usepackage{natbib,upgreek,mathtools}
\usepackage{rotating, caption}
%\usepackage{endfloat}
\usepackage{tgpagella}
\usepackage{xcolor}
\usepackage{hyperref}

%%%%%%%%%%%%%%%%%%%%%%%%%%%%%%%%%%%%%%%%%%%%%%%%%%%%%%%%%%%%
\definecolor{MyChange}{rgb}{1,0,0}
%%%%%%%%%%%%%%%%%%%%%%%%%%%%%%%%%%%%%%%%%%%%%%%%%%%%%%%%%%%%

%\usepackage[nolists]{endfloat}

%%%%%%%%%%%%%%%%%%%%%%%%%%%%%%%%%%%%%%%%%%%%%%%%%%%%%%%%%%%%
%Notation
%
\newcommand{\pbi}{\begin{itemize}}
	\newcommand{\pei}{\end{itemize}}
\newcommand{\pii}{\item}
%
\newcommand{\pbc}{\begin{center}}
	\newcommand{\pec}{\end{center}}
%
\newcommand{\pbe}{\begin{eqnarray*}}
	\newcommand{\pee}{\end{eqnarray*}}
%
\providecommand{\Pr}{\mathbb{Pr}} %real numbers
\let\hat=\widehat
\let\geq=\geqslant
\let\leq=\leqslant
%
\newcommand{\defeq}{\,\stackrel{{\rm \vartriangle}}{=}\,}
\newcommand{\pms}{\quad}
\newcommand{\simindep}{\,\stackrel{{\rm indep}}{\sim}\,}
\DeclareMathOperator*{\argmin}{arg\,\min}
%\newcommand{\med}{#1}{\,\stackrel{{\rm median}}{#1}\,}
%\newcommand\med[1]{\stackrel{{\rm median}}{#1}}
%
%\newcommand\med[1]{\stackrel{\rm median}{{\normalfont\mbox{#1}}}}
\newcommand\med[1]{\underset{#1}{\mathrm{med}}}

\providecommand{\Xd}{\dot{X}}
\providecommand{\zd}{\dot{z}}
\providecommand{\yd}{\dot{y}}
\providecommand{\xd}{\dot{x}}
\providecommand{\thetad}{\dot{\theta}}
\providecommand{\phid}{\dot{\phi}}
\providecommand{\nd}{{\dot{n}}}

\providecommand{\rhot}{\tilde{\rho}}
\providecommand{\sigmat}{\tilde{\sigma}}
\providecommand{\xit}{\tilde{\xi}}
\providecommand{\ut}{{\tilde{u}}}
\providecommand{\qt}{{\tilde{q}}}
\providecommand{\Qt}{{\tilde{Q}}}
\providecommand{\qb}{{\breve{q}}}
\providecommand{\Qb}{{\breve{Q}}}
%
\providecommand{\eps}{\epsilon}
\providecommand{\cvr}{{\theta,\phi}}
\providecommand{\cvrA}{{\theta \text{ and } \phi}}
\providecommand{\prm}{\alpha, \gamma, \sigma, \xi}
\providecommand{\prmA}{\alpha, \gamma, \sigma \text{ and } \xi}
\providecommand{\prmn}{\alpha, \gamma, \nu, \xi} %uses nu
\providecommand{\prmAn}{\alpha, \gamma, \nu \text{ and } \xi} %uses nu
%
\newcommand{\GP}{\mathrm{GP}}
\newcommand{\W}{\mathrm{Wbl}}
\newcommand{\TW}{\mathrm{TW}}
%
\newcommand{\un}[1]{\boldsymbol{#1}}
%
\providecommand{\np}{\vspace{10pt}}
%
\newcommand{\ed}[1]{\textcolor{red}{#1}}

%\onehalfspacing
\setlength{\parindent}{0cm}
\setlength{\parskip}{1em}

\providecommand{\np}{\vspace{10pt}}

%Page size
\oddsidemargin  -0.7in
\evensidemargin -0.7in
\textwidth      7.6in
\headheight     0.0in
\topmargin      -1.0in
\textheight     10.5in

\begin{document}
	
	\title{\textbf{Reply on first review} \np \\ {XXX} \np \\ by Leach et al.}
	\np
	\author{Please address all correspondence to Philip Jonathan (\emph{p.jonathan@lancaster.ac.uk})}
	\date{}
	\maketitle
	
	\section*{Summary}
	
	We thank four reviewers (referenced Reviewer 2-5) sincerely for their time and expertise in reviewing the manuscript. We note that both reviewers find merit in the work. 
	
	Reviewer 2 XXX. Reviewer 3 XXX. Reviewer 4 XXX. Reviewer 5 XXX.
	
	More generally, both reviewers make suggestions for improvement of the manuscript, which we reply to below, entailing some additions and modification of the manuscript. Where we feel the manuscript is already adequate on a point, we provide a note of explanation. 
	
	In the following sections, reviewer comments on the manuscript are given verbatim in italics and referenced by review section number `S', review reference number `R', comment number `C'. Our author replies (suffixed `A') are given in normal font per review comment. Larger changes to the manuscript are given in \ed{red} in both rejoinder and revised manuscript. References to locations in the article are made by us per section (e.g Section~4, S3.5), to avoid confusion with page- and line-number differences between manuscripts. 
	
	We hope that the revised manuscript is acceptable for publication, but would be happy of course to revise further in light of additional review comments.
	
	\section*{Review section 1. Are the objectives and the rationale of the study clearly stated?}
	\emph{S1.R2.C0 General comment. This study provides a detailed analysis of CMIP6 global coupled models, focusing on the effects of climate change at site-specific, regional, and global scales. The examination of changes in extreme quantiles, particularly the 100-year return value of climate variables, is highly relevant and enhances understanding of long-term trends. The methodology, combining extreme value analysis (GEV) and non-homogeneous Gaussian regression (NHGR), is well-suited for quantifying extreme events and spatial trends.The structure of the paper is clearly laid out, with a logical progression from data description and methodology to the presentation of results. The separation of global, climate zone, and specific location analyses provides a thorough exploration of the different scales at which climate change impacts may manifest. The inclusion of supplementary material further enhances the accessibility and understanding of the data and results. Only minor revisions are required to further enhance the clarity and impact of the paper. These include making a few refinements to the text for improved flow and precision, correcting typographical errors, and expanding the discussion of the results. Once these minor adjustments have been made, I believe the paper will be ready for publication. Overall, this work represents a significant contribution to the field, with well-defined objectives and a robust methodology. It offers valuable insights into the future projections of key climate variables and highlights the importance of considering both extremes and spatial averages when assessing climate change impacts.}
	
	S1.R2.C0.A1 We thank the reviewer for the positive comments. We have accommodated the majority of reviewer suggestions in the revised article, details below. XXX
		
	\emph{S1.R2.C1 Remark \#1 The introduction is thoughtfully organised, drawing on a range of references, including recent ones. The objectives are clearly articulated, and the article’s structure is comprehensively outlined. Furthermore, supplementary material is included to reinforce and support the content of the article. However, the data referenced in Leach (2024) is not available, as the provided link is not working. Please review and update the link in the reference “Leach, C., 2024. Changes over time in the 100-year return value of climate model variables: data. https://github.com/Callum-Leach/Climate-Change-100-Year-Return-Value-Offshore-Engineering-Data”, accordingly.}
	
	S1.R2.C1.A1 The hyperlink is correct, but without the double quotation at the end of the string. We have ensured that the double quotation mark is clearly separated from the link text. XXX
		
	\emph{S1.R2.C2 Remark \#2 In the first sentence of the 3rd paragraph of Section “2. Global coupled model output” (page 3 of the manuscript), it would be helpful to explicitly include the corresponding letters for initialisation, physics, and forcing of the model when describing the ensemble members. For improved clarity, you might consider revising the sentence as follows: “(…) we examine output for five climate model ensemble members (rX) where available; these correspond to a common initialisation (iX), physics (pX) and forcing (fX) per GCM, (…)”. This would provide readers with a clearer understanding of the specific configuration of each ensemble member, particularly when these details are referenced in Table 1.}
	
	S1.R2.C2.A1 We agree, and have added the description as suggested. XXX
	
	\emph{S1.R2.C3 Remark \#3 It would be beneficial to explicitly highlight why the 100-year return value is a critical metric for understanding climate change impacts. This could be incorporated into the introduction to ensure better alignment with the conclusions.}
	
	S1.R2.C3.A1 We agree, and have added a description as suggested. XXX

	\emph{S1.R2.C4 Remark \#4 It would be helpful to clarify how this study contributes to bridging gaps in the existing literature, thereby strengthening the rationale for the research. This could be highlighted in the introduction or the discussion section for better context.}
	
	S1.R2.C4.A1 We agree, and have added a description as suggested. XXX

	\emph{S1.R3.C1 yes}

	\emph{S1.R4.C1 yes}

	\emph{S1.R5.C1 Generally yes, however, the choices of variables and models needs to be put into context. Please see the Reviewer Comments to the Author Section.}
	
	S1.R5.C1.A1 Refer to S10.R5.CXXX.
	
	%Section 2%
	\section*{Review section 2. If applicable, is the application/theory/method/study reported in sufficient detail to allow for its replicability and/or reproducibility?}
	\emph{S2.R2.C1 Yes [X] No [] N/A []}
	
	S2.R2.C1.A1 XXX
		
	\emph{S2.R2.C1 Remark \#5 It would be constructive to clarify in Section “3. Methodology" the rationale or criteria used to select and present the results from the UKESM1-0-LL GCM in Section "SM5 Diagnostic plots for GEVR model fitting to global annual data, and NHGR modelling fitting to global means, for UKESM1-0-LL" of the Supplementary Material, as opposed to results from other models.}
	
	S2.R2.C1.A1 Thanks. We chose UKESM1-0-LL as a typical example. We have added text to this effect in SXXX of the revised manuscript. XXX
		
	\emph{S2.R2.C2 Remark \#6 It would be insightful to provide additional information about the specific parameters and settings used in the non-stationary GEV and NHGR models, as this would enable other researchers to replicate the statistical analysis.}
	
	S2.R2.C2.A1 XXX (Reference appendix and other sections where the model setup was discussed).
	
	\emph{S2.R3.C1 Yes [X] No [] N/A []}

	\emph{S2.R4.C1 Yes [X] No [] N/A []}

	\emph{S2.R5.C1 Yes [X] No [] N/A []}

	%Section 3%
	\section*{Review section 3. If applicable, are statistical analyses, controls, sampling mechanism, and statistical reporting (e.g., P-values, CIs, effect sizes) appropriate and well described?}

	\emph{S3.R2.C1 Yes [X] No [] N/A []}
	
	\emph{S3.R2.C2 Remark \#7 The paper discusses the choice of block maxima for the analysis of extreme values, but does not provide a clear justification for this selection over the peaks-over-threshold method. It would strengthen the manuscript to explicitly explain the reasoning behind selecting block maxima for the analysis, providing a more robust rationale for this approach.}
	
	S3.R2.C2.A1 XXX
	
	\emph{S3.R2.C3 Remark \#8 The paper would benefit from providing additional details regarding the criteria used to compare trends across models, particularly for parameters such as location ($\mu0$), scale ($\sigma0$), and shape ($\epsilon0$). A clearer explanation of these criteria would enhance the transparency and robustness of the analysis.}
	
	S3.R2.C3.A1 XXX
	
	\emph{S3.R3.C1 Yes [] No [X] N/A []}
	
	\emph{S3.R4.C1 Yes [X] No [] N/A []}
	
	\emph{S3.R5.C1 Yes [X] No [] N/A []}

	%Section 4%
	\section*{Review section 4. Could the manuscript benefit from additional tables or figures, or from improving or removing (some of the) existing ones?}

	\emph{S4.R2.C1 Remark \#9 The paper provides an adequate number of figures for the reader to fully comprehend the study. The Supplementary Material includes additional figures for further reference, and relevant results from the analysis of the graphs in the supplementary document are also discussed within the manuscript. The supplementary document is well-structured, with an introductory section that clearly links the different sections of the supplementary material to the figures in the main manuscript. The figures are well-organised and effectively present the corresponding results. However, for consistency, it would be beneficial if all the graphs in the figures within the Supplementary Material included the x-axis labels. For instance, the graphs in Sections SM1 (Figures SM1 and SM2) and SM4 (Figures SM19 and SM20) do not display the x-axis values, whereas all other graphs in the remaining sections do. Furthermore, in the captions of figures that include empty panels (Figures SM19 and SM20), it would be more coherent to add the note “Empty panels indicate that data for the specific combination of GCM and climate variable was not available for analysis,” as seen in the caption of Figures SM1 and SM2.}

	S4.R2.C1.A1 XXX

	\emph{S4.R2.C2 Remark \#10 In the fourth sentence of the 6th paragraph of Section “2.1. Compilation of spatial summaries: global and climate zone data” (page 5 of the manuscript), where it states “(…) Note also a suspect value in r1i1p1r1 for NorESM2-0-LL tas minimum for scenario SSP245 in the Antarctic (and hence Global), see Figures SM1 and SM8 for illustration: (…)”, it might be helpful to rephrase this as: “(…) see Figures SM8 and SM1, respectively, for illustration: (…)” in order to provide the reader with clearer guidance.}

	S4.R2.C2.A1 XXX

	\emph{S4.R2.C3 Remark \#11 In the first sentence of the 9th paragraph of Section “2.1. Compilation of spatial summaries: global and climate zone data” (page 6 of the manuscript), where it is stated, “Figures SM10-SM16 show corresponding global annual mean time-series for the seven GCMs in turn, for further comparison,” there appears to be a slight mistake. Figures SM10-SM16 actually pertain to “climate zone annual mean time-series,” rather than “global annual mean time-series.” I would suggest reviewing the entire paragraph and verifying the accuracy of the information presented.}

	S4.R2.C3.A1 XXX

	\emph{S4.R2.C4 Remark \#12 In the Supplementary Material, I suggest revising the caption for Figure SM25 (tas for UKESM1-0-LL) to clearly indicate that it represents global annual minima of tas for UKESM1-0-LL. The current caption is quite similar to that of Figure SM24 (UKESM1-0-LL, tas), which represents global annual maxima (a) and mean (b) values, potentially causing confusion for the reader.}

	S4.R2.C4.A1 XXX

	\emph{S4.R2.C5 Remark \#13 In the Supplementary Material, the captions for Figures SM17 and SM18 should include the clarification: “The first two characters of the GCM name are used for concise labelling” consistent with the explanation provided in Figure SM26.}

	S4.R2.C5.A1 XXX

	\emph{S4.R2.C6 Remark \#14 In the caption of Figure 2, it would be beneficial to include the explanation: “”Mxm” and “Mnm” in titles represent maximum and minimum, respectively” as was done in Figure 5 of the manuscript. This clarification is particularly important since these abbreviations appear here for the first time in a figure.}

	S4.R2.C6.A1 XXX

	\emph{S4.R2.C7 Remark \#15 Interpreting the means in the box-whisker plots presented in the Supplementary Material, Section SM6.1 (Figures SM26–29), is somewhat difficult; however, the presentation in the manuscript provides a clearer and more accessible format for interpretation.}

	S4.R2.C7.A1 XXX

	\emph{S4.R2.C8 Remark \#16 In the Supplementary Material, the caption for Table SM1 contains a discrepancy. Where it states “(…) thus we estimate an increase of 7.02K in the 100-year minimum tas in the Temperate North under scenario SSP585 over the next 100 years” the value for SSP585 should be 17.42K, not 7.02K. The value of 7.02K corresponds to SSP245. I kindly request that you revise it.}

	S4.R2.C8.A1 XXX

	\emph{S4.R2.C9 Remark \#17 In the caption of Table 4, it is stated that “(…) thus we estimate a reduction of 3.16 $ms^{-1}$ in the return value for sfcWind under scenario SSP585.” However, the climate zone and the scenario difference effect are not specified. I suggest providing this information for clarity.}

	S4.R2.C9.A1 XXX

	\emph{S4.R2.C10 Remark \#18 In the Supplementary Material, the caption for Table SM4 states that “(…) thus we estimate a reduction of 3.16 $ms^{-1}$ in the return value for sfcWind under scenario SSP585.” However, the value of “-3.16” does not appear in the table. I recommend reviewing and specifying the region to which this reduction applies.}

	S4.R2.C10.A1 XXX

	\emph{S4.R3.C1 nothing}

	\emph{S4.R4.C1 NO}

	\emph{S4.R5.C1 I believe this paper contains a sufficient number of figures and statistical analyses.}

	%Section 5%
	\section*{Review section 5. If applicable, are the interpretation of results and study conclusions supported by the data?}

	\emph{S5.R2.C1 Yes [X] No [] N/A []}

	\emph{S5.R2.C2 Remark \#19 The conclusions are generally well-supported by the data presented in the study. The interpretation of the results is consistent with the findings, and the study’s conclusions align with the key trends and patterns identified through the analysis. The authors effectively highlight the significance of their findings in relation to the broader context of climate change. No major revisions are needed in this section, as the conclusions accurately reflect the results. Overall, the study presents a clear and coherent interpretation that is well-supported by the data. However, to further strengthen the Section “5. Discussion and Conclusions”, the authors may wish to consider the remarks provided throughout this review, which suggest ways to enhance the clarity and robustness of the study’s interpretations and conclusions.}

	S5.R2.C2.A1 XXX

	\emph{S5.R3.C1 Yes [X] No [] N/A []}

	\emph{S5.R4.C1 Yes [X] No [] N/A []}

	\emph{S5.R5.C1 Yes [X] No [] N/A []}

	%Section 6%
	\section*{Review section 6. Have the authors clearly emphasized the strengths of their study/theory/methods/argument?}

	\emph{S6.R2.C0 The strengths of this study are well communicated, but the following suggestions could further emphasize them:}

	\emph{S6.R2.C1 Remark \#20 It would be beneficial to highlight the novel aspect of considering both extremes and spatial means across multiple models and scenarios, as this approach offers a more comprehensive view of climate variability.}

	S6.R2.C1.A1 XXX

	\emph{S6.R2.C2 Remark \#21 I suggest emphasising the contribution of using CMIP6 data to assess long-term trends in extreme values, highlighting how this study differs from those that focus on short-term calibration.}

	S6.R2.C2.A1 XXX

	\emph{S6.R3.C1 yes}

	\emph{S6.R4.C1 YES}

	\emph{S6.R5.C1 Yes, but please refer to the Reviewer Comments to the Author Section for further comments on this.}

	S6.R5.C1.A1 Refer to S10.R5.CXXX.

	%Section 7%
	\section*{Review section 7. Have the authors clearly stated the limitations of their study/theory/methods/argument?}

	\emph{S7.R2.C0 The authors acknowledge several limitations of the study, but the following additions could enhance transparency:}

	\emph{S7.R2.C1 Remark \#22 It would be helpful to clearly state that the absence of calibration for future extremes may lead to systematic biases, especially for variables with high spatio-temporal variability.}

	S7.R2.C1.A1 XXX

	\emph{S7.R2.C2 Remark \#23 It would be useful to discuss the potential impact of using annual maxima, as opposed to alternative methods such as peaks-over-threshold, on the robustness of the findings.}

	S7.R2.C2.A1 XXX

	\emph{S7.R3.C1 yes}

	\emph{S7.R4.C1 YES}

	\emph{S7.R5.C1 There are some limitations of the study that the author could mention in the conclusion section. See the Reviewer Comments to the Author Section.}

	S7.R5.C1.A1 XXX

	%Section 8%
	\section*{Review section 8. Does the manuscript structure, flow or writing need improving (e.g., the addition of subheadings, shortening of text, reorganization of sections, or moving details from one section to another)?}

	\emph{S8.R2.C0 The manuscript is well-organized, and the structure and flow are generally clear and logical. Each section is appropriately defined, and the progression of ideas from one to another is smooth. The use of subheadings aids in navigation, and the presentation of content is cohesive. No significant changes or improvements are needed in terms of structure. Overall, the manuscript is well-structured and effectively communicates the research.}

	\emph{S8.R3.C0 no}

	\emph{S8.R4.C0 NO}

	\emph{S8.R5.C0 The manuscript is well written.}

	%Section 9%
	\section*{Review section 9. Could the manuscript benefit from language editing?}

	\emph{S9.R2.C0 No}

	\emph{S9.R3.C0 Yes}

	\emph{S9.R4.C0 No}

	\emph{S9.R5.C0 No}

	%Section 10%
	\section*{Review section 10. This field is optional. If you have any additional suggestions beyond those relevant to the questions above, please number and list them here.}

	\emph{S10.R2.C1 Remark \#25 The current submission contains six highlights, whereas the journal specifies a maximum of five bullet points. I recommend combining or removing some of these points to comply with this requirement. Additionally, three of the highlights (1st, 4th and 6th) exceed the 85-character limit, including spaces. I suggest rephrasing them to make them more concise while maintaining their key messages. Finally, in the first highlight, the phrase "We quantify (…)" is written in the first person. It would be more appropriate to use an impersonal structure, such as "Quantification of (...)", for consistency.}

	S10.R2.C1.A1 XXX

	\emph{S10.R2.C2 Remark \#26 The abstract provides a clear overview of the purpose of the research, the principal results, and the major conclusions, as required by the journal. It outlines the aim of assessing changes in tail characteristics for wind, solar irradiance, and temperature variables using CMIP6 model outputs. The key methods employed, such as non-stationary extreme value models and non-homogeneous Gaussian regression, are well described. Additionally, the abstract highlights the main findings, including the weak evidence for changes in wind extremes and the stronger evidence for changes in solar irradiance and temperature. However, the current abstract exceeds the journal's word limit of 200 words. I recommend revising it to ensure conciseness while retaining the essential information on the study's purpose, methods, key results, and conclusions.}

	S10.R2.C2.A1 XXX

	\emph{S10.R2.C3 Remark \#27 The manuscript includes nine keywords, which exceeds the journal's specified limit of seven. I recommend reducing the number of keywords to comply with the journal's guidelines. Additionally, the final keyword ends with a semicolon (";"), which should be removed to align with proper formatting. Please revise the keywords accordingly.}

	S10.R2.C3.A1 XXX

	\emph{S10.R2.C4 Remark \#28 I have identified a typo in Section "1. Introduction". In the second sentence of the 1st paragraph (page 1 of the manuscript), the phrase "(…) that global mean sea level has increases (…)" should be corrected to "(…) that global mean sea level has increased (…)."}

	S10.R2.C4.A1 XXX

	\emph{S10.R2.C5 Remark \#29 I have identified a typo in Section "2.1. Compilation of spatial summaries: global and climate zone data". In the fifth sentence of the 3rd paragraph (page 4 of the manuscript), the phrase "(…) this figure also provides illustrations of generally reducing trends in global annual maximum rsds, and increasing trends in global annuam minimum of tas." should be corrected to "(…) and increasing trends in global annual minimum of tas."}

	S10.R2.C5.A1 XXX

	\emph{S10.R2.C6 Remark \#30 I have identified a typo in Section "2.2. North Atlantic and Celtic Sea data". In the first sentence of the 2nd paragraph (page 7 of the manuscript), where it states "(…) and given in Figures SM19-20." it should be "(…) are given in Figures SM19-20." Kindly confirm.}

	S10.R2.C6.A1 XXX

	\emph{S10.R3.C0 In this paper, the effects of climate change on extreme climate variables are studied, focusing on the changing trends of short-wave radiation (rsds), near-surface wind speed (sfcWind), maximum daily wind speed (SFCwind Max) and near-surface air temperature (tas). The output of several global coupled climate models (GCMs) and a variety of different climate scenarios (SSP126, SSP245, SSP585) were statistically quantified. The non-stationary generalized extreme value regression model (GEVR) and non-homogeneous Gaussian regression model (NHGR) were used to evaluate the extreme value and annual mean change of climate variables, respectively. At the same time, Markov chain Monte Carlo method (MCMC) was used to quantify the model uncertainty. This paper is innovative in that it uses seven climate models and multiple climate variables to comprehensively assess the impact on extreme climate. This comprehensive multi-model and multi-variable analysis method improves the breadth and comparability of the research, especially in the quantification of extreme value distribution. In addition, non-stationary generalized extreme value regression (GEVR) and non-uniform Gaussian regression (NHGR) models are used to analyze the changes of extreme value and mean value, and the Bayesian method is combined to estimate the changes of extreme value in 100 years, which improves the reliability of the results to a certain extent. However, there are still limitations in the article, and it is suggested to be revised and published:}

	\emph{S10.R3.C1 (1) It has been repeatedly mentioned that there are significant differences in the forecast results between different models, especially in wind speed and extreme value changes. This shows that there are still significant uncertainties when different climate models simulate the same variables. It is recommended that the authors further explore ways to reduce model inconsistencies, such as through model calibration or higher-resolution simulations.}

	S10.R3.C1.A1 XXX

	\emph{S10.R3.C2 (2) A more in-depth physical discussion of the weaker trends in some variables, such as MRI-ESM2-0 wind speed maxima, to explain whether these phenomena are related to the limitations of climate models or the atmospheric dynamics of some specific regions.}

	S10.R3.C2.A1 XXX

	\emph{S10.R3.C3 (3) The conclusion of the paper does not mention whether the research problem in the introduction has been solved, so it is suggested to improve it.}

	S10.R3.C3.A1 XXX

	\emph{S10.R3.C4 (4) Although the colors currently used (such as green, orange and gray) distinguish different climate scenarios, the colors are not bright enough, and high contrast color schemes can be considered, such as red and blue.}

	S10.R3.C4.A1 XXX

	\emph{S10.R3.C5 (5) The description of the model and data in the paper is rather redundant. For example, the model source and variable definition of CMIP6 are mentioned several times in different parts, and it is suggested to simplify.}

	S10.R3.C5.A1 XXX

	\emph{S10.R4.C0 In this paper, the changes of the tail characteristics of wind speed, solar radiation and temperature output of CMIP6 climate model due to climate forcing are studied with time. To assess changes in centennial reproducible values of wind speed, solar radiation, and temperature variables in the CMIP6 climate model output, as well as changes in annual mean data for these variables over the period 2015 to 2100. The authors used Bayesian inference to estimate the parameters of a non-stationary extreme value model (GEVR) and a heterogeneous Gaussian regression (NHGR) model. Using these models, the authors quantified centennial recurrence changes in annual extremes and changes in annual mean levels over the period 2025 to 2125. The study considered three different climate scenarios (SSP126, SSP245, SSP585) and multiple climate model ensemble members. It is found that for wind speed variables, the study results show that the centennial recurrence value changes with time and climate scenarios are weak. In contrast, solar radiation and temperature variables show more pronounced changes in centennial recurrence values. For annual averages, the evidence for changes in wind speed variables over time is stronger, especially in the Northern Hemisphere, but the magnitude of change is smaller. The suggestions are as follows:}

	\emph{S10.R4.C1 (1) Although Bayesian analysis does not rely on P-values, providing the median, mean, and confidence intervals (such as confidence intervals) of the posterior distribution will help readers understand the uncertainty of parameter estimates.}

	S10.R4.C1.A1 XXX

	\emph{S10.R4.C2 (2) The specific application of non-stationary extreme value model and heterogeneous Gaussian regression model and the reasons for parameter selection can be further elaborated to help readers better understand the applicability and limitations of the model.}

	S10.R4.C2.A1 XXX

	\emph{S10.R4.C3 (3) The limitations of their research should be more clearly stated in the discussion section, including limitations in model selection, data availability, analytical methods, and interpretation of results.}

	S10.R4.C3.A1 XXX

	\emph{S10.R4.C4 (4) In the discussion section, the possible reasons for the uncertainty of the hundred-year recurrence value of the wind speed variable are discussed in detail, and the potential impact on the offshore engineering design is discussed.}

	S10.R4.C4.A1 XXX

	\emph{S10.R4.C5 (5) It is recommended that the authors further explore in the discussion section how these findings can be applied to actual ocean engineering design, especially in consideration of the effects of climate change.}

	S10.R4.C5.A1 XXX

	\emph{S10.R5.C0 The paper presents a statistical analysis of projected future changes in both extremes and means of four climate variables, based on outputs from seven CMIP6 GCMs, under three climate scenarios, and multiple ensemble runs. The primary objective is to explore the tail characteristics of the distributions for wind, solar irradiance, and temperature across different spatial scales: globally, within specific climate zones, and at select point locations in the North Atlantic and Celtic Sea, which I assume are of particular interest to Shell. I find the work to be thorough and of significant relevance to the scientific community. The authors' statistical approach is detailed and meticulous, demonstrating a clear mastery of the subject matter. Given the importance of the topic and the rigorous methodology employed, I recommend this paper for publication, subject to addressing a few minor points. My main comments are outlined below.}

	\emph{S10.R5.C1 (1) The authors do not provide any rationale for selecting the climate variables, but given that this is a scientific paper, it is valuable to provide context in the introduction. Please include some background information and perhaps specific examples explaining why wind, solar irradiance, and temperature were chosen. Are these variables relevant for floating solar installations? Are they crucial for offshore wind farms? What are the reasons behind these selections? Additionally, offer a brief explanation of why these specific seven GCMs were selected, even if it is simply for convenience because they were readily available in your database. Overall, both the choice of variables and the selection of models would benefit from more background information.}

	S10.R5.C1.A1 XXX

	\emph{S10.R5.C2 (2) If I understand correctly, for sfcWind, rsds, and tas, you extracted daily averages. While I recognize that this choice simplifies the analysis, it raises questions about its impact on extreme statistics. This potential limitation is not addressed anywhere in the text. It would be beneficial to mention this as a limitation of the study in the conclusion. Is this the reason why you selected sfcWindmax for the wind?}

	S10.R5.C2.A1 XXX

	\emph{S10.R5.C3 (3) Figure 3 is unclear. It's not immediately apparent how the ensemble members are represented, as I only see a single point per model in most of the subplots and the colours for the scenarios. I believe this figure requires revision for better clarity.}

	S10.R5.C3.A1 XXX

	\emph{S10.R5.C4 (4) The study assumes that all three model parameters (location, scale, and shape) change linearly over time. Have you tested the impact of assuming changes in only the location parameter, for example? Additionally, do you have any plans to incorporate covariates, such as the North Atlantic Oscillation Index, for specific regions? Addressing these points could strengthen the robustness of the analysis.}

	S10.R5.C4.A1 XXX

	\emph{S10.R5.C5 (5) The final statistics tables provide a concise and engaging way to summarize the overall variability in the analysis and the confidence in the projected changes. However, the assumption that all models contribute equally is a bit of a stretch. There is no discussion in the text regarding the independence of GCMs, and treating ensemble runs as independent realisations when they essentially represent the same climate model with different initial conditions, model set up, or test runs, is also a questionable assumption. I recommend adding a dedicated discussion in the conclusion section to address model independence and explicitly state these assumptions as a limitation. Additionally, while it may be clear to the authors, the methodology for calculating the expected value of change and the probability of change should be explicitly described. For example, is the expected value simply the mean of all return values across the GCMs and ensemble runs? Is the probability of change calculated as the fraction of models where $\Delta > 0$ over the total number of models? Clarifying these details would greatly enhance the transparency of the analysis.}

	S10.R5.C5.A1 XXX

	%DO WE WANT TO KEEP THIS LINE%
	
	\emph{S10.R5.C8 \ed{DO WE WANT TO KEEP THIS TEXT} Minor comments: It is quite challenging to give detailed comments on the text without line numbers, but I will try my best. Next time I suggest submitting with line numbers.}

	\emph{S10.R5.C9 In the objective and outlines subsection " For the global and climate zone analyses, we think it IS interesting …" Eq. 3: do you weigh your climate zone averages by latitude?}

	S10.R5.C9.A1 XXX

	\emph{S10.R5.C10 Figure 2: It might be actually more interesting to show here the ensemble averages from all your model runs, instead of just one model.}

	S10.R5.C10.A1 XXX

	\emph{S10.R5.C11 Page 6 "In summary, climate scenario effects on wind speed variables are less pronounced than on rsds and tas, with the exception of the Arctic zone; physically, perhaps this is related to the occurrence of polar lows and cyclonic systems there, but it IS not clear…"}

	S10.R5.C11.A1 XXX

	\emph{S10.R5.C12 Page 6, just before Figure 3: "Section 3" remove the s.}

	S10.R5.C12.A1 XXX



	%\pagebreak
	\bibliographystyle{elsarticle-harv}
	\bibliography{phil}
	%\bibliography{C:/Philip/LaTeX/Bibliography/phil}
	%\bibliography{C:/Users/Philip.Jonathan/PhilipGit/Code/LaTeX/phil}
	%\bibliography{C:/Philip/Git/Cod/LaTeX/phil}
	%\bibliography{phil}
	
\end{document}